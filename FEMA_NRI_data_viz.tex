% Options for packages loaded elsewhere
% Options for packages loaded elsewhere
\PassOptionsToPackage{unicode}{hyperref}
\PassOptionsToPackage{hyphens}{url}
\PassOptionsToPackage{dvipsnames,svgnames,x11names}{xcolor}
%
\documentclass[
  letterpaper,
  DIV=11,
  numbers=noendperiod]{scrartcl}
\usepackage{xcolor}
\usepackage{amsmath,amssymb}
\setcounter{secnumdepth}{-\maxdimen} % remove section numbering
\usepackage{iftex}
\ifPDFTeX
  \usepackage[T1]{fontenc}
  \usepackage[utf8]{inputenc}
  \usepackage{textcomp} % provide euro and other symbols
\else % if luatex or xetex
  \usepackage{unicode-math} % this also loads fontspec
  \defaultfontfeatures{Scale=MatchLowercase}
  \defaultfontfeatures[\rmfamily]{Ligatures=TeX,Scale=1}
\fi
\usepackage{lmodern}
\ifPDFTeX\else
  % xetex/luatex font selection
\fi
% Use upquote if available, for straight quotes in verbatim environments
\IfFileExists{upquote.sty}{\usepackage{upquote}}{}
\IfFileExists{microtype.sty}{% use microtype if available
  \usepackage[]{microtype}
  \UseMicrotypeSet[protrusion]{basicmath} % disable protrusion for tt fonts
}{}
\makeatletter
\@ifundefined{KOMAClassName}{% if non-KOMA class
  \IfFileExists{parskip.sty}{%
    \usepackage{parskip}
  }{% else
    \setlength{\parindent}{0pt}
    \setlength{\parskip}{6pt plus 2pt minus 1pt}}
}{% if KOMA class
  \KOMAoptions{parskip=half}}
\makeatother
% Make \paragraph and \subparagraph free-standing
\makeatletter
\ifx\paragraph\undefined\else
  \let\oldparagraph\paragraph
  \renewcommand{\paragraph}{
    \@ifstar
      \xxxParagraphStar
      \xxxParagraphNoStar
  }
  \newcommand{\xxxParagraphStar}[1]{\oldparagraph*{#1}\mbox{}}
  \newcommand{\xxxParagraphNoStar}[1]{\oldparagraph{#1}\mbox{}}
\fi
\ifx\subparagraph\undefined\else
  \let\oldsubparagraph\subparagraph
  \renewcommand{\subparagraph}{
    \@ifstar
      \xxxSubParagraphStar
      \xxxSubParagraphNoStar
  }
  \newcommand{\xxxSubParagraphStar}[1]{\oldsubparagraph*{#1}\mbox{}}
  \newcommand{\xxxSubParagraphNoStar}[1]{\oldsubparagraph{#1}\mbox{}}
\fi
\makeatother

\usepackage{color}
\usepackage{fancyvrb}
\newcommand{\VerbBar}{|}
\newcommand{\VERB}{\Verb[commandchars=\\\{\}]}
\DefineVerbatimEnvironment{Highlighting}{Verbatim}{commandchars=\\\{\}}
% Add ',fontsize=\small' for more characters per line
\usepackage{framed}
\definecolor{shadecolor}{RGB}{241,243,245}
\newenvironment{Shaded}{\begin{snugshade}}{\end{snugshade}}
\newcommand{\AlertTok}[1]{\textcolor[rgb]{0.68,0.00,0.00}{#1}}
\newcommand{\AnnotationTok}[1]{\textcolor[rgb]{0.37,0.37,0.37}{#1}}
\newcommand{\AttributeTok}[1]{\textcolor[rgb]{0.40,0.45,0.13}{#1}}
\newcommand{\BaseNTok}[1]{\textcolor[rgb]{0.68,0.00,0.00}{#1}}
\newcommand{\BuiltInTok}[1]{\textcolor[rgb]{0.00,0.23,0.31}{#1}}
\newcommand{\CharTok}[1]{\textcolor[rgb]{0.13,0.47,0.30}{#1}}
\newcommand{\CommentTok}[1]{\textcolor[rgb]{0.37,0.37,0.37}{#1}}
\newcommand{\CommentVarTok}[1]{\textcolor[rgb]{0.37,0.37,0.37}{\textit{#1}}}
\newcommand{\ConstantTok}[1]{\textcolor[rgb]{0.56,0.35,0.01}{#1}}
\newcommand{\ControlFlowTok}[1]{\textcolor[rgb]{0.00,0.23,0.31}{\textbf{#1}}}
\newcommand{\DataTypeTok}[1]{\textcolor[rgb]{0.68,0.00,0.00}{#1}}
\newcommand{\DecValTok}[1]{\textcolor[rgb]{0.68,0.00,0.00}{#1}}
\newcommand{\DocumentationTok}[1]{\textcolor[rgb]{0.37,0.37,0.37}{\textit{#1}}}
\newcommand{\ErrorTok}[1]{\textcolor[rgb]{0.68,0.00,0.00}{#1}}
\newcommand{\ExtensionTok}[1]{\textcolor[rgb]{0.00,0.23,0.31}{#1}}
\newcommand{\FloatTok}[1]{\textcolor[rgb]{0.68,0.00,0.00}{#1}}
\newcommand{\FunctionTok}[1]{\textcolor[rgb]{0.28,0.35,0.67}{#1}}
\newcommand{\ImportTok}[1]{\textcolor[rgb]{0.00,0.46,0.62}{#1}}
\newcommand{\InformationTok}[1]{\textcolor[rgb]{0.37,0.37,0.37}{#1}}
\newcommand{\KeywordTok}[1]{\textcolor[rgb]{0.00,0.23,0.31}{\textbf{#1}}}
\newcommand{\NormalTok}[1]{\textcolor[rgb]{0.00,0.23,0.31}{#1}}
\newcommand{\OperatorTok}[1]{\textcolor[rgb]{0.37,0.37,0.37}{#1}}
\newcommand{\OtherTok}[1]{\textcolor[rgb]{0.00,0.23,0.31}{#1}}
\newcommand{\PreprocessorTok}[1]{\textcolor[rgb]{0.68,0.00,0.00}{#1}}
\newcommand{\RegionMarkerTok}[1]{\textcolor[rgb]{0.00,0.23,0.31}{#1}}
\newcommand{\SpecialCharTok}[1]{\textcolor[rgb]{0.37,0.37,0.37}{#1}}
\newcommand{\SpecialStringTok}[1]{\textcolor[rgb]{0.13,0.47,0.30}{#1}}
\newcommand{\StringTok}[1]{\textcolor[rgb]{0.13,0.47,0.30}{#1}}
\newcommand{\VariableTok}[1]{\textcolor[rgb]{0.07,0.07,0.07}{#1}}
\newcommand{\VerbatimStringTok}[1]{\textcolor[rgb]{0.13,0.47,0.30}{#1}}
\newcommand{\WarningTok}[1]{\textcolor[rgb]{0.37,0.37,0.37}{\textit{#1}}}

\usepackage{longtable,booktabs,array}
\usepackage{calc} % for calculating minipage widths
% Correct order of tables after \paragraph or \subparagraph
\usepackage{etoolbox}
\makeatletter
\patchcmd\longtable{\par}{\if@noskipsec\mbox{}\fi\par}{}{}
\makeatother
% Allow footnotes in longtable head/foot
\IfFileExists{footnotehyper.sty}{\usepackage{footnotehyper}}{\usepackage{footnote}}
\makesavenoteenv{longtable}
\usepackage{graphicx}
\makeatletter
\newsavebox\pandoc@box
\newcommand*\pandocbounded[1]{% scales image to fit in text height/width
  \sbox\pandoc@box{#1}%
  \Gscale@div\@tempa{\textheight}{\dimexpr\ht\pandoc@box+\dp\pandoc@box\relax}%
  \Gscale@div\@tempb{\linewidth}{\wd\pandoc@box}%
  \ifdim\@tempb\p@<\@tempa\p@\let\@tempa\@tempb\fi% select the smaller of both
  \ifdim\@tempa\p@<\p@\scalebox{\@tempa}{\usebox\pandoc@box}%
  \else\usebox{\pandoc@box}%
  \fi%
}
% Set default figure placement to htbp
\def\fps@figure{htbp}
\makeatother





\setlength{\emergencystretch}{3em} % prevent overfull lines

\providecommand{\tightlist}{%
  \setlength{\itemsep}{0pt}\setlength{\parskip}{0pt}}



 


\KOMAoption{captions}{tableheading}
\makeatletter
\@ifpackageloaded{caption}{}{\usepackage{caption}}
\AtBeginDocument{%
\ifdefined\contentsname
  \renewcommand*\contentsname{Table of contents}
\else
  \newcommand\contentsname{Table of contents}
\fi
\ifdefined\listfigurename
  \renewcommand*\listfigurename{List of Figures}
\else
  \newcommand\listfigurename{List of Figures}
\fi
\ifdefined\listtablename
  \renewcommand*\listtablename{List of Tables}
\else
  \newcommand\listtablename{List of Tables}
\fi
\ifdefined\figurename
  \renewcommand*\figurename{Figure}
\else
  \newcommand\figurename{Figure}
\fi
\ifdefined\tablename
  \renewcommand*\tablename{Table}
\else
  \newcommand\tablename{Table}
\fi
}
\@ifpackageloaded{float}{}{\usepackage{float}}
\floatstyle{ruled}
\@ifundefined{c@chapter}{\newfloat{codelisting}{h}{lop}}{\newfloat{codelisting}{h}{lop}[chapter]}
\floatname{codelisting}{Listing}
\newcommand*\listoflistings{\listof{codelisting}{List of Listings}}
\makeatother
\makeatletter
\makeatother
\makeatletter
\@ifpackageloaded{caption}{}{\usepackage{caption}}
\@ifpackageloaded{subcaption}{}{\usepackage{subcaption}}
\makeatother
\usepackage{bookmark}
\IfFileExists{xurl.sty}{\usepackage{xurl}}{} % add URL line breaks if available
\urlstyle{same}
\hypersetup{
  pdftitle={EDS 230 - NRI Visualization},
  pdfauthor={Henry Oliver},
  colorlinks=true,
  linkcolor={blue},
  filecolor={Maroon},
  citecolor={Blue},
  urlcolor={Blue},
  pdfcreator={LaTeX via pandoc}}


\title{EDS 230 - NRI Visualization}
\author{Henry Oliver}
\date{}
\begin{document}
\maketitle


\subsubsection{Introduction:}\label{introduction}

The purpose of this analysis is to compare National Risk Index (NRI)
scores for California counties to those in other states across the US.
This analysis utilizes the National Risk Index (NRI) Counties feature
layer provided by FEMA. The NRI Counties feature layer contains
county-level data for the Risk Index, Expected Annual Loss, Social
Vulnerability, and Community Resilience.The National Risk Index data
helps to illustrate the communities most at risk for 18 natural hazards
across the United States and territories: avalanche, coastal flooding,
cold wave, drought, earthquake, hail, heat wave, hurricane, ice storm,
inland flooding, landslide, lightning, strong wind, tornado, tsunami,
volcanic activity, wildfire, and winter weather. The National Risk Index
data provides Risk Index values, scores and ratings based on data for
Expected Annual Loss due to natural hazards, Social Vulnerability, and
Community Resilience. Separate values, scores and ratings are also
provided for Expected Annual Loss, Social Vulnerability, and Community
Resilience.{[}1{]}

\begin{Shaded}
\begin{Highlighting}[]
\FunctionTok{library}\NormalTok{(tidyverse)}
\FunctionTok{library}\NormalTok{(janitor)}
\FunctionTok{library}\NormalTok{(here)}
\end{Highlighting}
\end{Shaded}

\subsubsection{Read-In Data}\label{read-in-data}

\begin{Shaded}
\begin{Highlighting}[]
\NormalTok{nri\_raw }\OtherTok{\textless{}{-}} \FunctionTok{read\_csv}\NormalTok{(}\FunctionTok{here}\NormalTok{(}\StringTok{"data/National\_Risk\_Index\_Counties\_807384124455672111.csv"}\NormalTok{), }\AttributeTok{show\_col\_types =} \ConstantTok{FALSE}\NormalTok{) }\SpecialCharTok{\%\textgreater{}\%} 
  \FunctionTok{clean\_names}\NormalTok{()}
\end{Highlighting}
\end{Shaded}

\paragraph{``How do FEMA National Risk Index scores for counties in
California compare to those in other
states?''}\label{how-do-fema-national-risk-index-scores-for-counties-in-california-compare-to-those-in-other-states}

\subsubsection{Variables of interest:}\label{variables-of-interest}

object\_id, state\_name, state\_name\_abbreviation, county\_name,
county\_fips\_code, population\_2020, building\_value,
agricultural\_value, national\_risk\_index\_score\_composite,
expected\_annual\_loss\_score\_composite.

The final National Risk Index (NRI) Composite Risk Score represents a
nationally standardized measure of natural hazard risk that integrates
multi-hazard exposure, community vulnerability, and community
resilience. The score is calculated as the product of aggregated hazard
risk and social vulnerability, adjusted by resilience as a mitigating
factor, and reflects relative risk compared to other U.S. counties
rather than absolute probability or expected loss.

I'm interested in seeing how the composite NRI Scores compare to those
of other states. Specifically, I want to see how California ranks among
states with the highest ratio of high NRI counties to low NRI counties.

\begin{Shaded}
\begin{Highlighting}[]
\CommentTok{\# Select columns I\textquotesingle{}m interested in}
\NormalTok{nri\_clean }\OtherTok{\textless{}{-}}\NormalTok{ nri\_raw }\SpecialCharTok{\%\textgreater{}\%} 
  \FunctionTok{select}\NormalTok{(}\StringTok{"objectid"}\NormalTok{, }\StringTok{"state\_name"}\NormalTok{, }\StringTok{"state\_name\_abbreviation"}\NormalTok{, }\StringTok{"county\_name"}\NormalTok{, }\StringTok{"county\_fips\_code"}\NormalTok{, }\StringTok{"national\_risk\_index\_score\_composite"}\NormalTok{,)}
\end{Highlighting}
\end{Shaded}

For my visualization, I want to see how many of California counties are
in the top 25 percent of NRI scorse nationwide. I also want to see how
this compares to other states. Since every state has a different number
of counties, I'm going to view this data as a proportion of
highNRI:lowNRI counties per state. Also, I only want to see the states
with the highest scores, otherwise my visualization will get too busy.

\subsubsection{Plot}\label{plot}

\begin{Shaded}
\begin{Highlighting}[]
\CommentTok{\# Set a threshold for the top 25\% of scores for NRI}
\NormalTok{threshold }\OtherTok{\textless{}{-}} \FunctionTok{quantile}\NormalTok{(nri\_clean}\SpecialCharTok{$}\NormalTok{national\_risk\_index\_score\_composite, }\FloatTok{0.75}\NormalTok{, }\AttributeTok{na.rm =} \ConstantTok{TRUE}\NormalTok{)}

\CommentTok{\# Make a dataframe of continental US states that have the highest ratio of counties with 25th percentile to counties in bottom 75th percentile.}
\NormalTok{nri\_state }\OtherTok{\textless{}{-}}\NormalTok{ nri\_clean }\SpecialCharTok{\%\textgreater{}\%} 
  \FunctionTok{group\_by}\NormalTok{(state\_name\_abbreviation) }\SpecialCharTok{\%\textgreater{}\%} 
  \FunctionTok{filter}\NormalTok{(}\SpecialCharTok{!}\NormalTok{state\_name\_abbreviation }\SpecialCharTok{\%in\%} \FunctionTok{c}\NormalTok{(}\StringTok{"GU"}\NormalTok{, }\StringTok{"VI"}\NormalTok{, }\StringTok{"PR"}\NormalTok{, }\StringTok{"AS"}\NormalTok{, }\StringTok{"MP"}\NormalTok{, }\StringTok{"DC"}\NormalTok{, }\StringTok{"DE"}\NormalTok{)) }\SpecialCharTok{\%\textgreater{}\%} 
  \FunctionTok{summarise}\NormalTok{(}\AttributeTok{proportion\_top\_25\_pct =} \FunctionTok{mean}\NormalTok{(national\_risk\_index\_score\_composite }\SpecialCharTok{\textgreater{}=}\NormalTok{ threshold)) }\SpecialCharTok{\%\textgreater{}\%} 
  \FunctionTok{arrange}\NormalTok{(}\FunctionTok{desc}\NormalTok{(proportion\_top\_25\_pct))}

\CommentTok{\# get top 10 states}
\NormalTok{nri\_top10 }\OtherTok{\textless{}{-}}\NormalTok{ nri\_state }\SpecialCharTok{\%\textgreater{}\%}
  \FunctionTok{slice\_head}\NormalTok{(}\AttributeTok{n =} \DecValTok{10}\NormalTok{)}
\FunctionTok{ggplot}\NormalTok{(nri\_top10, }\FunctionTok{aes}\NormalTok{(}\AttributeTok{x =} \FunctionTok{reorder}\NormalTok{(state\_name\_abbreviation, }\SpecialCharTok{{-}}\NormalTok{proportion\_top\_25\_pct), }\AttributeTok{y =}\NormalTok{ proportion\_top\_25\_pct }\SpecialCharTok{*} \DecValTok{100}\NormalTok{)) }\SpecialCharTok{+}
  \FunctionTok{geom\_col}\NormalTok{(}\AttributeTok{fill =} \StringTok{"\#F5F5F5"}\NormalTok{) }\SpecialCharTok{+} \CommentTok{\# white bars}
  \FunctionTok{geom\_text}\NormalTok{(}\FunctionTok{aes}\NormalTok{(}\AttributeTok{label =}\NormalTok{ state\_name\_abbreviation), }\AttributeTok{vjust =} \SpecialCharTok{{-}}\FloatTok{0.5}\NormalTok{, }\AttributeTok{color =} \StringTok{"\#F5F5F5"}\NormalTok{, }\AttributeTok{size =} \FloatTok{3.5}\NormalTok{) }\SpecialCharTok{+} \CommentTok{\# state labels above bars}
  \FunctionTok{coord\_cartesian}\NormalTok{(}\AttributeTok{ylim =} \FunctionTok{c}\NormalTok{(}\DecValTok{50}\NormalTok{, }\DecValTok{105}\NormalTok{)) }\SpecialCharTok{+} \CommentTok{\# zoom y{-}axis to start at 50\%}
  \FunctionTok{scale\_y\_continuous}\NormalTok{(}
    \AttributeTok{breaks =} \FunctionTok{seq}\NormalTok{(}\DecValTok{50}\NormalTok{, }\DecValTok{100}\NormalTok{, }\DecValTok{10}\NormalTok{), }\CommentTok{\# tick marks every 10\%}
    \AttributeTok{labels =} \FunctionTok{paste0}\NormalTok{(}\FunctionTok{seq}\NormalTok{(}\DecValTok{50}\NormalTok{, }\DecValTok{100}\NormalTok{, }\DecValTok{10}\NormalTok{), }\StringTok{"\%"}\NormalTok{), }\CommentTok{\# add \% symbol}
    \AttributeTok{expand =} \FunctionTok{c}\NormalTok{(}\DecValTok{0}\NormalTok{, }\DecValTok{0}\NormalTok{) }\CommentTok{\# no padding}
\NormalTok{  ) }\SpecialCharTok{+}
  \FunctionTok{labs}\NormalTok{(}
    \AttributeTok{title =} \StringTok{"US States with Highest Proportion of High{-}Risk Counties"}\NormalTok{,}
    \AttributeTok{subtitle =} \StringTok{"Natural hazard exposure concentrated in western and coastal states"}\NormalTok{,}
    \AttributeTok{y =} \StringTok{"Proportion of counties with}\SpecialCharTok{\textbackslash{}n}\StringTok{75th percentile NRI score"}\NormalTok{,}
    \AttributeTok{x =} \ConstantTok{NULL}\NormalTok{,}
    \AttributeTok{caption =} \StringTok{"* Excludes Washington DC (one county) and Delaware (three counties)}\SpecialCharTok{\textbackslash{}n}\StringTok{Data: FEMA National Risk Index (2025 Release)"}
\NormalTok{  ) }\SpecialCharTok{+}
  \FunctionTok{theme\_minimal}\NormalTok{() }\SpecialCharTok{+}
  \FunctionTok{theme}\NormalTok{(}
    \AttributeTok{plot.background =} \FunctionTok{element\_rect}\NormalTok{(}\AttributeTok{fill =} \StringTok{"\#C41E3A"}\NormalTok{, }\AttributeTok{color =} \ConstantTok{NA}\NormalTok{), }\CommentTok{\# red background}
    \AttributeTok{panel.background =} \FunctionTok{element\_rect}\NormalTok{(}\AttributeTok{fill =} \StringTok{"\#C41E3A"}\NormalTok{, }\AttributeTok{color =} \ConstantTok{NA}\NormalTok{), }\CommentTok{\# red panel}
    \AttributeTok{panel.grid.major.x =} \FunctionTok{element\_blank}\NormalTok{(), }\CommentTok{\# remove vertical gridlines}
    \AttributeTok{panel.grid.minor =} \FunctionTok{element\_blank}\NormalTok{(), }\CommentTok{\# remove minor gridlines}
    \AttributeTok{panel.grid.major.y =} \FunctionTok{element\_line}\NormalTok{(}\AttributeTok{color =} \StringTok{"\#F5F5F560"}\NormalTok{, }\AttributeTok{linewidth =} \FloatTok{0.3}\NormalTok{), }\CommentTok{\# transparent white horizontal gridlines}
    \AttributeTok{axis.text =} \FunctionTok{element\_text}\NormalTok{(}\AttributeTok{color =} \StringTok{"\#F5F5F5"}\NormalTok{), }\CommentTok{\# white axis text}
    \AttributeTok{axis.title.y =} \FunctionTok{element\_text}\NormalTok{(}\AttributeTok{color =} \StringTok{"\#F5F5F5"}\NormalTok{, }\AttributeTok{margin =} \FunctionTok{margin}\NormalTok{(}\AttributeTok{r =} \DecValTok{10}\NormalTok{)), }\CommentTok{\# white y{-}axis title}
    \AttributeTok{axis.text.x =} \FunctionTok{element\_blank}\NormalTok{(), }\CommentTok{\# hide x{-}axis labels}
    \AttributeTok{plot.title =} \FunctionTok{element\_text}\NormalTok{(}\AttributeTok{color =} \StringTok{"\#F5F5F5"}\NormalTok{, }\AttributeTok{size =} \DecValTok{14}\NormalTok{, }\AttributeTok{face =} \StringTok{"bold"}\NormalTok{, }\AttributeTok{margin =} \FunctionTok{margin}\NormalTok{(}\AttributeTok{b =} \DecValTok{10}\NormalTok{)), }\CommentTok{\# white bold title}
    \AttributeTok{plot.subtitle =} \FunctionTok{element\_text}\NormalTok{(}\AttributeTok{color =} \StringTok{"\#F5F5F5"}\NormalTok{, }\AttributeTok{size =} \DecValTok{11}\NormalTok{, }\AttributeTok{margin =} \FunctionTok{margin}\NormalTok{(}\AttributeTok{b =} \DecValTok{0}\NormalTok{)), }\CommentTok{\# white subtitle}
    \AttributeTok{plot.caption =} \FunctionTok{element\_text}\NormalTok{(}\AttributeTok{color =} \StringTok{"\#F5F5F5"}\NormalTok{, }\AttributeTok{hjust =} \DecValTok{0}\NormalTok{, }\AttributeTok{size =} \DecValTok{8}\NormalTok{, }\AttributeTok{margin =} \FunctionTok{margin}\NormalTok{(}\AttributeTok{t =} \DecValTok{10}\NormalTok{)), }\CommentTok{\# white caption left{-}aligned}
    \AttributeTok{plot.margin =} \FunctionTok{margin}\NormalTok{(}\AttributeTok{t =} \DecValTok{20}\NormalTok{, }\AttributeTok{r =} \DecValTok{10}\NormalTok{, }\AttributeTok{b =} \DecValTok{10}\NormalTok{, }\AttributeTok{l =} \DecValTok{10}\NormalTok{) }\CommentTok{\# add top margin for label space}
\NormalTok{  )}
\end{Highlighting}
\end{Shaded}

\pandocbounded{\includegraphics[keepaspectratio]{FEMA_NRI_data_viz_files/figure-pdf/unnamed-chunk-4-1.pdf}}

\subsubsection{Questions}\label{questions}

\textbf{1. What are your variables of interest and what kinds of data
(e.g.~numeric, categorical, ordered, etc.) are they (a bullet point list
is fine)?} state\_name, state\_name\_abbreviation, county\_name,
county\_fips\_code, national\_risk\_index\_score\_composite.

The risk score composite is numeric, state and county names are
categorical.

\textbf{2. How did you decide which type of graphic form was best suited
for answering the question? What alternative graphic forms could you
have used instead? Why did you settle on this particular graphic form?}

I just wanted something that would make it easy to compare similar
states to each other directly, and see a clear ranking. I thought about
using a bubble chart or similar chart to see areas compared to one
another on a larger spread (this would also be able to potentially show
more US states), but I was really only intersted in the states with the
highest NRI scores, and I think theres elegance in the simplicity of a
bar graph.

\textbf{3. Summarize your main finding in no more than two sentences.}

California has the second highest proportion of counties with NRI scores
in the 75th percentile. There is a steep dropoff in the proportion of
above to below 75th percentile at about 80\% in Hawaii.

\textbf{4. What modifications did you make to this visualization to make
it more easily readable?}

Changing the colors to contrast effectively, and removing unnecessary
plot ink link axes, and a plot box. I made my major gridlines slightly
more transparent than my data labels so that the data labels would stand
out even when the two overlap. Also specifying margins so no text was
overlapping or crowded.

\textbf{5. Is there anything you wanted to implement, but didn't know
how? If so, please describe.} I would have liked to somehow indicate the
number of counties per state, or the total area that is occupied by
counties that are in the 75th percentile of NRI scores - But I couldn't
figure out how to add that without busying up the graph. Just adding
another number on each bar wouldn't have been that meaningful and would
have made the figure look significantly worse. I couldn't think of a
creative and visually appealing way to incorporate another metric,
population, land area, or number of counties. In my experimentation this
one just seemed to be the most meaningful and useful.

\subsubsection{References}\label{references}

{[}1{]}https://www.fema.gov/flood-maps/products-tools/national-risk-index

https://eds-240-data-viz.github.io/course-materials/assignments/HW2.html




\end{document}
